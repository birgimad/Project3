\chapter{Introduction}
%An introduction where you explain the aims and rationale for the physics case and what you have done. At the end of the introduction you should give a brief summary of the structure of the report

The project mainly discusses about various integration methods namely Gaussian quadrature methods and Monte carlo methods that can be employed to solve an integral function. The integral we are interested in is the quantum mechanical expectation value of the correlation energy between two electrons which repel each other via the classical Coulomb interaction. 
In the Gaussian quadrature method two important orthogonal polynomials Legendre and Laguerre are used to evaluate the integral. Since Legendre polynomial is defined in the interval $[-1,1]$, solving the integral gave unsatisfactory results. To improve the results the integral which was in cartesian coordinate is converted to spherical coordinates with different integration limits and Laguerre polynomial is introduced which is defined in the interval $[0,\infty]$. 
A comparison of the results using the two different polynomial is made. 

In order to employ Monte Carlo methods we assumed the system can be described as a probability density function and random numbers were generated to cover uniformly in the interval [0,1]. 
First a brute force approach using cartesian coordinates is used to find the solution then the method is improved with importance sampling in which the coordinates are changed to spherical coordinates as in the Laguerre method with the radial component being exponentially distributed. 
An error estimation is made in both cases. A comparison of all the method that we used to solve the integral is done in \secref{sec:ComputationalTime}.

This report mainly consists of two sections.  First section gives a short introduction on the nature of the problem and then move on to a detailed description of both the Gaussian quadrature and Monte Carlo methods for solving the integral. \chapref{chap:Results} is for the discussion and interpretation of results.