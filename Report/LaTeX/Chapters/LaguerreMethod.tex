\section{Laguerre Method for Computing the Integral}
\label{sec:LaguerreMethod}
In the Gauss-Laguerre method integrals of the form
\begin{align}
	I = \int _0 ^{\infty } x^{\alpha } e^{-x} g(x) dx
	\label{eq:LaguerreMethod1}
\end{align}
can be solved using Laguerre polynomials and a weight function given by
\begin{align}
	W(x) = x^{\alpha} e^{-x}
	\label{eq:LaguerreMethod2}
\end{align} 
Hence, if the variables of the integral in 
\matref{eq:NatureOfTheProblem2} 
are changed from cartesian coordinates to spherical coordinates with 
$r_i \in [0;\infty )$, 
$\theta_i \in [0;\pi]$, and 
$\phi_i \in [0;2\pi]$, the integral can be solved by using the roots and corresponding weights of Legendre polynomials for $\theta_i$ and $\phi_i$ and use roots and corresponding weights of Laguerre polynomials for $r_i$.
The algorithm for finding the roots and their corresponding weights of the Laguerre polynomials will, however, not be discussed here.
An important note is, though, that since the weight function includes $x^{\alpha}$ and $e^{-x}$, the function that is to be evaluated in the roots of the $n$'th Laguerre polynomial is 
\begin{align}  
   g(r_1, r_2, \theta_1, \theta_2, \phi_1, \phi_2 ) = 
   \frac{e^{-2\alpha (r_1 + r_2 ) + r_1 + r_2}}{\sqrt{r_1^2 + r_2 ^2 -2 r_1 r_2 cos(\beta)}}  
sin(\theta_1 ) sin(\theta_2 )    
   \label{eq:LaguerreMethod3}
\end{align} 
with $cos(\beta)$ described by \matref{eq:CartesianSpherical4}.
