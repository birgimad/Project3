\subsection{Integral Written in Cartesian and Spherical Coordinates}
\label{subsec:CartesianSpherical}
Writing out \matref{eq:NatureOfTheProblem2} in cartesian coordinates in which
\begin{align}
	\v{r}_i = x_i \hat{\v{i}} + y_i \hat{\v{j}} + z_i \hat{\v{k}}
	\label{eq:CartesianSpherical1}
\end{align}
with $\hat{\v{i}}$, $\hat{\v{j}}$ and $\hat{\v{k}}$ being unit vectors in the x-, y- and z-direction, respectively and $x_i, y_i, z_i, \in (-\infty ; \infty )$, yields
\begin{align}
	\left< \frac{1}{| \v{r}_1 - \v{r}_2 |} \right> 
   = \int _{-\infty } ^{\infty }   
   \frac{e^{-2\alpha \left( \sqrt{x_1^2+y_1^2+z_1^2} + \sqrt{x_2^2+y_2^2+z_2^2}\right) }}{\sqrt{(x_1-x_2)^2+(y_1-y_2)^2+(z_1-z_2)^2}} d x_1 d y_1 d z_1 d x_2 d y_2 d z_2
   \label{eq:CartesianSpherical2}
\end{align}
with the integral being from $-\infty$ to $\infty$ over all six variables.
This representation of the integral will be used in the Gauss-Legendre method for computing the integral.
The following lines of code shows how to compute the function to be integrated in c++.
\begin{lstlisting}
double int_function(double x1, double x2, double x3, double y1, double y2, double y3)
{
  int alpha = 2.0;
  double denominator = pow((x1-y1),2)+pow((x2-y2),2)+pow((x3-y3),2);
  double exponent = exp(-2.0*alpha*(sqrt(x1*x1+x2*x2+x3*x3)+sqrt(y1*y1+y2*y2+y3*y3)));
  if (denominator < pow(10,-6)){return 0;}
  else return exponent/sqrt(denominator);
}
\end{lstlisting}
A problem arises if the denominator becomes very small, when computing the fraction since this would cause the function value to become very large. 
To avoid this problem, the function value is set equal to zero (that is $10^{-6}$ if the denominator becomes too small.
\fxnote{is this an okay explanation??} 

For the Laguerre method, spherical coordinates with 
$r_i \in [0;\infty )$, 
$\theta_i \in [0;\pi]$, and 
$\phi_i \in [0;2\pi]$ are used.
When writing the considered integral in spherical coordinates, the problem to solve becomes 
\begin{align}
	\left< \frac{1}{| \v{r}_1 - \v{r}_2 |} \right> 
   = &\int  
   \frac{e^{-2\alpha (r_1 + r_2 )}}{\sqrt{r_1^2 + r_2 ^2 -2 r_1 r_2 cos(\beta)}}  
r_1 ^2 r_2 ^2 sin(\theta_1 ) sin(\theta_2 ) 
	dr_1 dr_2 d\theta_1 d\theta_2 d\phi_1 d\phi_2    
   \label{eq:CartesianSpherical3}
\end{align}
with the integral having the limits as described above.
$cos(\beta )$ is then given by
\begin{align}
	cos(\beta) = cos(\theta_1) cos(\theta_2) + sin(\theta_1) sin(\theta_2 ) cos(\phi_1 - \phi_2 )
	\label{eq:CartesianSpherical4}
\end{align}